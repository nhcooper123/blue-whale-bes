% Preamble
\documentclass[a4paper,12pt]{article}

\usepackage[osf]{mathpazo} % palatino
%\usepackage{ms1}            % load the template
%\usepackage[round]{natbib} % author-year citations
\usepackage[superscript,biblabel]{cite} % for superscript citations
\usepackage{graphicx}
\usepackage{parskip} 
\usepackage{amsmath}
\usepackage{textcomp} % for parts per mille symbol  
\usepackage{pdflscape} % landscape   
\pagenumbering{arabic}    
\linespread{1.66}



% -------------------------------------------------------------------------------------------------------
% some custom commands

% Add support for highlighting
\usepackage{color}
\newcommand{\hilight}[1]{\colorbox{yellow}{#1}}

% figure numbering override
\renewcommand*{\thefigure}{S\arabic{figure}} % make Fig S1 not Fig 1
\renewcommand*{\thetable}{S\arabic{table}} % make Table S1 not Table 1

% -------------------------------------------------------------------------------------------------------

% Title page information
\title{Supplementary Information from:\\
\textit{Reconstructing the last known movements of one of Nature's giants}}
% 90 characters max

\author{Natalie Cooper, Andrew L. Jackson, Katharyn S. Chadwick,\\ 
Ellen J. Coombs, Richard C. Sabin and Clive N. Trueman}

\date{}


%\vfill


% End of preamble

\begin{document}
%\title{Supplementary Information - Cooper et al. 2017}
%\modulolinenumbers[1]   % Line numbering on every line

\maketitle

\parindent = 1.5em
\addtolength{\parskip}{.3em}


\section*{Supplementary Methods}
 
\subsection*{CHARACTERISATION OF THE MEASURED BALEEN}
The baleen plate yielded 78 discrete samples of baleen. 
Across the baleen sample, $\delta^{15}$N values show clear cycles with periods of relatively high $\delta^{15}$N values interspersed by relatively short periods where $\delta^{15}$N values are c.1\text{\textperthousand} lower. 
Fourier analysis (periodogram function in the R package TSA\cite{Chan:2012aa}) revealed a strong periodic repetition with a 13.3cm periodicity. 
$\delta^{13}$C values also show repeated fluctuations in the first (oldest) 60cm of the baleen plate (behavioural phase one), with a similar periodicity of 13.5cm (Fig. \ref{figs1}). 
A distinct change in the isotopic pattern along the baleen plate occurs at around 60cm behavioural phase two), with a transition to relatively positive and constant $\delta^{13}$C values, associated with a weakening of the periodic fluctuation seen in $\delta^{15}$N values, and an overall reduction in $\delta^{15}$N values.

% figure s1
\begin{figure}[!htbp]
  \centering
  \includegraphics[width = \linewidth]{figures/Figure-S1-periodograms.png}
  \caption{Fourier analysis of $delta^{15}$N (A,B) and $delta^{13}$C (C,D) values of the full (A,C) and behavioral phase one only (B,D) components of the baleen plate. Inset values are the two highest frequency spectral densities (in cm). Note the baleen plate is 80cm long. Periodic fluctuation with a distance of 13.5cm along the baleen plate is consistently recovered, but is weaker or absent behavioural phase two} % check legend
  \label{figs1}
\end{figure}
 
Cross-correlation analysis demonstrates a strong negative covariance between $\delta^{13}$C and $\delta^{15}$N values within behavioral phase one (oldest 60cm of the baleen plate; Fig. \ref{figs2}), but no relationship between $\delta^{13}$C and $\delta^{15}$N values exists during behavioral phase two (the final 20cm of the baleen plate). 

% figure s2
\begin{figure}[!htbp]
  \centering
  \includegraphics[width = \linewidth]{figures/Figure-S2-cross-cor.png}
  \caption{Cross correlation between $\delta^{13}$C and $\delta^{15}$N values for behavioural phase one showing strong negative covariance with a periodicity of around 13cm} % check legend
  \label{figs2}
\end{figure}
 
\subsection*{BASELINE ISOTOPE COMPARISONS}
Isotope-enabled biogeochemical ocean models\cite{magozzi2017using} (Schmittner and Somes) were used to characterize the isotopic composition of phytoplankton expected in different potential foraging grounds. 
Annual average $\delta^{15}$N POM values were provided by C.J. Somes (pers.comm) based on a 5$^{\circ}$ resolution biogeochemical model. 
$\delta^{13}$C POM values were simulated using an isotopic extension to the NEMO-MEDUSA model\cite{magozzi2017using}.
 
Simulated $\delta^{15}$N POM values are relatively positive in the northeast Atlantic north of c.60$^{\circ}$N, and relatively negative in the central and southern North Atlantic, associated with high levels of nitrogen fixation. 
Annual average $\delta^{13}$C POM values largely vary with latitude, with more negative values in more northerly regions. 
In the central North Atlantic, $\delta^{13}$C POM values are relatively positive in the west, reflecting warm gulf stream waters (Fig. \ref{figs3}). 
The temporally coincident periodic, inverse fluctuations in $\delta^{13}$C and $\delta^{15}$N values in the initial phase of the baleen plate are therefore consistent with migration between waters north of 60$^{\circ}$N and more southerly southern waters in the North Atlantic, with northerly occupancy in summer months and southerly occupancy in winter months. 
Incorporation of carbon and nitrogen with isotopic ratios consistent with southerly latitudes during winter months indicates continued feeding during migration and winter residency.
 
\begin{figure}[!htbp]
  \centering
  \includegraphics[width = \linewidth]{figures/Figure-S3-plankton-d13C-map.png}
  \caption{Isoscape map of the $delta^{13}$C values from POM used as simulated food in the movement model.}
  \label{figs3} 
\end{figure}

The isotopic composition of carbon in phytoplankton also varies through seasons as isotopic fractionation of carbon during photosynthesis is strongly influenced by sea surface temperature (Laws et al 1995). 
Thus temporal variations in $\delta^{13}$C POM values are superimposed on latitudinal gradients. 
The scale and nature of temporal variation in $\delta^{13}$C POM values also varies with latitude, with higher latitude seas showing greater intra-annual variation in $\delta^{13}$C POM values linked to strongly seasonal phytoplankton growth dynamics. 
We therefore used monthly simulated $\delta^{13}$C POM values to simulate the isotopic expression of phytoplankton expected to be encountered by whales exhibiting differing movement behaviours.
 
\subsection*{ISOTOPE MODEL SIMULATIONS}
We initially simulated temporal variations in baseline (phytoplankton) isotopic compositions of that would be encountered by whales foraging within broad geographic regions (Norwegian Sea, west UK/Irish shelf, Canaries/west Azores, Mid Atlantic ridge; Fig. \ref{figs4}). 
Strong seasonal dynamics in $\delta^{13}$C values are evident in all northerly regions, characterized by a rapid increase to annual maximum $\delta^{13}$C values associated with the onset of the spring phytoplankton bloom, followed by more gradual decline in $\delta^{13}$C values towards minima in winter conditions (Fig. \ref{figs4}). 
In temperate latitudes around the British Isles, seasonal dynamic cycles are present, but dampened (Fig. \ref{figs4}), whereas in subtropical regions exemplified by the Canaries and western Azores/Mid Atlantic ridge areas, $\delta^{13}$C values are relatively positive and constant throughout the year (Fig. \ref{figs4}).
 
  \begin{figure}[!htbp]
    \centering
      \includegraphics[width=\linewidth]{figures/Figure-S4-facet-wrap-d13C.png}
      \caption{Daily $\delta^{13}$C POM values in regions occupied by 30 simulated blue whales foraging over a two year cycle. \hilight{OR} Alternative $\delta^{13}$C series arising from resident-only movement models restricting region to Norway, Ireland, Canaries and the Mid Atlantic ridge as labelled over a simulated 2-year period with daily resolution and 30 replicate simulations per region.  } % check legend
      \label{figs4}
  \end{figure}
 
\subsubsection*{Matching isotope model simulations with whale isotopes in behavioural phase one}
Temporal dynamics in $\delta^{13}$C values simulated in northerly latitudes are similar to the measured $\delta^{13}$C profile during the four years of behavioral phase one. 
Simulated profiles display the same asymmetric isotopic cycles associated with development of the spring bloom, with annual $\delta^{13}$C maxima in June (Fig. \ref{figs4}, Fig. 1). 
Simulated temporal isotopic profiles for northern resident whales, do not display clear isotopic minima preceding the onset of the bloom. 
Adding seasonal north-south migrations within mid-high latitude regions to foraging models, closely matches the measured profile, including $\delta^{13}$C minima preceding the June bloom (Fig. \ref{figs4}, Fig. 1).
 
\subsubsection*{Matching isotope model simulations with whale isotopes in behavioural phase two}
Behavioural phase two is characterized by a dramatic shift to maximum $\delta^{13}$C values in June of 1889 followed by approximately 18 months of relatively positive and constant $\delta^{13}$C values. 
$\delta^{15}$N values in this phase are relatively low and the isotopic cyclicity associated with north-south migration is absent in 1890. 
The final sample representing feeding shortly before death in Irish waters implies a return to isotopic conditions more consistent with behavioural phase one. 
The relatively positive and constant carbon isotope composition of behavioural phase two is consistent with residency (and continued feeding) in more southerly latitudes, particularly in the relatively isotopically positive waters southwest of the Azores.
 
Accordingly we modelled whale movements allowing five years of seasonal north-south migration in the north east northern Atlantic, followed by rapid migration to subtropical waters in the spring of 1890, relative residency for eight months and rapid migration returning to more northerly waters in the spring of 1891.
The resulting simulated profiles closely match the measured baleen plate, capturing all major temporal fluctuations (Fig. 1). 
Consistently low $\delta^{15}$N values in 1890 are also consistent with residency in relatively southern waters. 
 
The dramatic positive excursion in $\delta^{13}$C values seen between March and July 1889 cannot be replicated purely through movement to warmer waters, thus we tentatively ascribe this signal to the physiological demands of pregnancy.  
A captive feeding study in south American sea lions\cite{cardona2017temporal} showed positive $\delta^{13}$C values associated with pregnancy, falling at birth, and a gradual increase in $\delta^{15}$N  values during lactation. 
$\delta^{15}$N values in whale baleen during the autumn of 1890 show a gradual increase, consistent with the isotopic expression associated with lactation in south American fur seals\cite{cardona2017temporal}.
 
Blue whales have a gestation period of 10-12 months, and calves suckle for approximately 6-7 months\cite{handbook}. 
The proposed timescale of pregnancy around early spring 1889, birth in spring 1890, lactation through summer and autumn 1890 and return to northern feeding grounds in early 1891 is therefore fully consistent with the assumed reproductive ecology of blue whales.
 
\subsection*{AGENT-BASED WHALE MOVEMENT MODEL}
We simulated the likely isotopic expression associated with feeding and movement behaviours by building an agent based model sampling $\delta^{13}$C POM values predicted by the isotopic extension to the NEMO-MEDUSA model\cite{magozzi2017using,yool2013medusa}, where movement behavior is influenced by SST and phytoplankton biomass estimates provided by NEMO-MEDUSA\cite{yool2013medusa}, and bathymetry (Fig. \ref{figs5}).
 
The agent-based model developed employs a simple set of movement rules dictated by behavioural state which is itself fixed according to calendar month by the operator. 
At each daily step, the probability of movement, movement directions and distance traveled are sampled from probability distributions to allow individual variation.

%  \begin{figure}[!htbp]
%    \centering
%      \includegraphics[width=12cm]{XXX.pdf}
%      \caption{Summary diagram of agent-based movement model structure.}
%      \label{figs5}
%  \end{figure}
 
Initial boundary conditions are defined with a maximum temperature of 25$^{\circ}$C and minimum temperature of 3$^{\circ}$C. The likelihood of movement (i.e. whether to move or not from the current location) is sampled from a binomial distribution with the probability of movement influenced by behavioural state and external conditions. The maximum daily movement distance permitted in each behavioural state is defined as a random sample of a Gaussian distribution with specified mean and standard deviation (see Table \ref{tables1}).
 
The behavioural states employed in the model simulations are foraging, migrating north, migrating south and nursing. When foraging, the probability of movement and likely direction of movement is primarily controlled by the biomass of phytoplankton in the current and adjacent 1$^{\circ}$ cells. When migrating, a fixed higher probability of movement is given to southerly or northerly movement directions, and the probability of movement is influenced by the ambient sea surface temperature and phytoplankton biomass. When nursing a calf, movement is limited to a daily mean of 25km, as long as sea surface temperatures remain between 18$^{\circ}$C and 25$^{\circ}$C.
 
Parameters defined for the agent-based model best matching the measured profile are summarised in Table \ref{tables1}.

\newpage

\begin{landscape}

\centering
\begin{table}
  \begin{tabular}{|p{5cm}|c|c|c|c|} 
    \hline
    & \multicolumn{4}{|c|}{\textbf{Behavioural state}} \\
    \hline
    \textbf{Parameters} & \textbf{Foraging} & \textbf{Migrating north} & \textbf{Migrating south} & \textbf{Nursing}\\
    \hline
    Month - behavioural phase one (1460 days) & May - Oct & Feb - April & Nov - Jan & NA\\
    \hline
    Month - pregnancy (365 days) & June - Oct & April - May & Nov - March & NA\\
    \hline
    Month - nursing (240 days) & NA & NA & NA & Month???\\
    \hline
    Minimum sea surface temperature ($^{\circ}$C) & 3 & 3 & 3 & 18\\
    \hline
    Maximum sea surface temperature ($^{\circ}$C) & 25 & 25 & 25 & 25\\
    \hline
    Maximum daily distance traveled (km) & $N(\mu=30, \sigma=10)$ & $N(\mu=150, \sigma=10)$ & $N(\mu=150, \sigma=10)$ & $N(\mu=20, \sigma=5)$\\
    \hline
    Plankton biomass threshold for remaining to forage (units) & 50 & NA & NA & NA\\
    \hline
  \end{tabular}
  \caption{Model parameters for agent-based model of whale movement.}
  \label{tables1}
\end{table}

\end{landscape}
 
% References
\newpage
\bibliographystyle{nature}
\bibliography{blue-whale}

\end{document}