% Preamble
\documentclass[a4paper,12pt]{article}

\usepackage[osf]{mathpazo} % palatino
\usepackage{ms}            % load the template
\usepackage[round]{natbib} % author-year citations
\usepackage{graphicx}
\usepackage{parskip}      
\pagenumbering{arabic}    
\linespread{1.66}

% Title page information
\title{Reconstructing the last known movements of one of Nature's giants}
% 90 characters max

\author{
  Natalie Cooper$^{1}$ Jackson2, Trueman3 and Kate3, Coombs1,4, Sabin1 in some order
}
\date{}
\affiliation{\noindent{\footnotesize
  $^1$ Department of Life Sciences, Natural History Museum London, Cromwell Road, London, SW7 5BD, UK. natalie.cooper@nhm.ac.uk. Fax: +44 1 677 8094; Tel: +44 1 896 5083.\\ 
  $^2$ National Oceanographic Centre, University of Southampton, Southampton, UK.\\
  $^3$ School of Natural Sciences, Trinity College Dublin, Dublin 2, Ireland. a.jackson@tcd.ie. \\
  $^4$ UCL\\
}}

\vfill

%\runninghead{}
%\keywords{}\\

% End of preamble

\begin{document}
\modulolinenumbers[1]   % Line numbering on every line

\mstitlepage

\parindent = 1.5em
\addtolength{\parskip}{.3em}
% Abstract 200 -300 words max
\section{Abstract}

The blue whale, *Balaenoptera musculus*, is the largest animal to have ever lived, and one of the most heavily influenced by human actions. 
Over 50 years since the International Whaling Commission banned hunting of blue whales, North Atlantic stocks are  still only at \% of pre-whaling levels.
To conserve this endangered species we need to understand its distribution, behaviour and ecology, however, high resolution data on whale ecology are scarce, particularly for the North Atlantic population. 
Here we use baleen samples from the Natural History Museum's iconic blue whale, along with new movement models, to reconstruct the movements of the whale over the last three years of its life.
We show that the whale migrated rather than remaining resident in the northern Atlantic year round. 
This is the first clear quantitative evidence of historical migrations in the species

Museum collections provide a unique opportunity to infer animal movements with non invasive methods.


\section{Main text}
% 1500 words max (excludes refs title, authors, abstract, acknowledgements)

In 1891, a 25m long female blue whale beached off the coast of Wexford in Ireland. In 2017, her skeleton will go on display in the Natural History Museum's (NHM) rennovated Hintze Hall, making her the most famous whale in the UK, if not the world. Ironically, we know very little about this animal, or the species as a whole. 

Using stable isotpes (more detail) and a novel model of whale movement that incorporates X, Y and Z (see SI) we were able to trace the most likely movement history of the NHM blue whale. The models do this. We compared these to the carbon traces found for the blue whale (Fig 1C)

This basic lack of understanding is even scarier

Whaling was an intense pressure for whales during this period. The first blue whale fisheries began in XXX, but by 1900 fisheries had moved outside the area because stocks were so depleted. Before whaling, blue whale densities in the North Atlantic are estinated at X-X (depending on the methods used), but by 1900 it is likely that only a few hundred whales were left. The NHM blue whale thus represents an individual from a species at the brink of extinction. Blue whales were the first species that humans legislated to save, and represent a big conservation success story, with numbers of some populations appearing to recover to pre whaling levels.

Even today, high resolution behavioural information for large baleen whales (Balaenopteridae), and cetaceans in general, is rare, particularly for populations in the North Atlantic (by contrast, Pacific populations are more extensively studied, and patches of the North Atlantic are well studied, for example off Iceland)

Our findings have implications beyond this individual whale, and beyond cetaceans. Our models allow us  to ID migrating vs resident pops. Can be used for otehr stuff from whales to seals to sharks. immensely valuable.

\section{Methods (condensed)}
%<3000 words

\subsection{Data Availability}

\section{References}

\section{Supplementary Information}

\section{Acknowledgements}
This work was funded by the British Ecological Society (grant: 5771/6815). 
We thank XXX for comments on drafts.

\section{Author contributions}

\section{Author information}

\section{Tables}

\section{Figure legends}

\bibliographystyle{mee2}
\bibliography{darkside}

\newpage
\section{Figures}

%  \begin{figure}[!htbp]
%    \centering
%      \includegraphics[width=12cm]{XXX.pdf}
%      \caption{}
%      \label{}
%  \end{figure}

\end{document}